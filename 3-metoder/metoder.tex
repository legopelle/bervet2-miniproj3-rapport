% !TeX spellcheck = sv_SE
\section{Numeriska metoder}
\label{sec:metoder}

Vi kan lösa matrisekvationen \ref{eqn:ls} direkt i \textsc{matlab}, men när polynomgraden växer, kommer matrisen vi inverterar att bli mer och mer instabil. Detta kan motverkas genom att ortogonalisera den, vilket görs i \textsc{matlab}:s \lstinline|polyfit|. Våra mätvärden är också täta, vilket försvårar minsta-kvadratanpassningen. Men även det kan motverkas genom att centrera datan innan anpassning, vilket görs automatiskt av \lstinline|polyfit|.

Vår data behöver behandlas eftersom vissa datapunkter saknas. Dessa utmärks med att värdet är 0. Vi vill därför ta ut de nollskilda elementen med \lstinline|nz = find(data ~= 0)| vilket ger deras index. Vi har då vektorer med $x$ och $y$ värden för de datapunkter där $ y \neq 0$. Då gör vi en minsta-kvadratanpassning genom \lstinline|p = polyfit(x, y, M)| för polynom av grad \lstinline|M|.

Vi kan då avtrendifiera datan genom att subtrahera trenden (det anpassade polynomet) från datan \lstinline|detrend = y - polyval(p,x)|. Sedan lägger vi tillbaka de avsaknade mätvärdena på samma sätt som innan och tar medelvärdet av varje månad. Koden nedan itererar först över varje månad, och sedan var 12:e element i datan. Den summerar i \lstinline|sum_months| om värdet är skiljt från 0. En räknare \lstinline|numval| håller reda på antalet termer i summan. 
\begin{lstlisting}
for i=1:12
    for j=i:12:N
        if (mx_det(j) ~= 0)
            sum_months(i) = sum_months(i) + mx_det(j);
            numvals(i) = numvals(i) + 1;
        end
    end
end

months_avg = months ./ numval;
\end{lstlisting}

\subsection{Val av polynomgrad}

Vi kan implementera ekvation \ref{eqn:kaw} som 
\begin{lstlisting}
p = polyfit(x, y, N);
res = y - polyval(p, x);
sr = sum(res.^2);
kaw = sr/(M - N - 1);
\end{lstlisting}
för polynom av grad \lstinline|N| med antal datapunkter \lstinline|M|.