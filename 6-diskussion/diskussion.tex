% !TeX spellcheck = sv_SE
\section{Diskussion}
\label{sec:diskussion}

Om vi undersöker topparna i intervallet $t \in [0, 200h]$ så kan vi uppskatta periodtiden. För $\delta_R = 0.2$ ger detta ungefär 25h och för $\delta_R = 0.08$ ungefär 50h. Detta är rimligt då $\delta_R$ är proportionell mot R-proteinets nedbrytning. Om den bryts ner mer sällan avtar den långsammare och perioderna blir längre.

I \textsc{miniprojekt 1} mättes $D_A$ och $D_R$ med koncentrationer (egentligen antal i enhetsvolymen), medan vi nu mätte direkt i antal. Koncentrationen antogs vara kontinuerlig så att ett differentialekvationssystem var rimligt. I detta projekt arbetade vi istället med diskreta antal och sannolikheter för enskilda reaktioner.