% !TeX spellcheck = sv_SE
\section{Inledning}
\label{sec:inledning}

Dikväveoxid, \ce{N2O}, är en mycket potent växthusgas. Den är som molekyl per molekyl 200 gånger så stark som koldioxid. Det är därför av största intresse att kartlägga dess beteende i atmosfären. I denna rapport kommer vi undersöka om mängden \ce{N2O} i atmosfären är säsongsberoende.

Vi använder data från \emph{National Oceanic and Atmospheric
Administration – Climate Monitoring and Diagnostics
Laboratory (NOAA-CMDL) Global Cooperative Air
Sampling Network} från 1977 till 2000. Datan avtrendifieras med en polynomiell minsta-kvadratanpassning och undersöks sedan månadsvis.